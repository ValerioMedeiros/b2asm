
%\documentclass[gray,trans]{beamer} % com o mod transpar�ncia
%\documentclass[gray]{beamer}
\documentclass[trans,note]{beamer}

\usetheme{Madrid}

%\usetheme{Marburg} % Template ideial para SBMF
%\usetheme{CambridgeUS}
 %\usetheme{Boadilla}
%\usetheme{default}


%\usecolortheme{structure} 
%\usebeamercolor{structure}

% \usefonttheme[onlylarge]{structurebold}
% \usefonttheme{default}
% \setbeamerfont*{frametitle}{size=\normalsize,series=\bfseries}
 % \setbeamertemplate{navigation symbols}{}

% \usepackage[latin1]{inputenc}
% \usepackage[brazil]{babel}
\usepackage[english]{babel}
\usepackage[latin1]{inputenc}

\usepackage{graphicx,url}
\usepackage{fancyhdr}

\usepackage{tikz}
\usetikzlibrary{arrows}
 \tikzstyle{block}=[draw opacity=0.3,line width=1.0cm]

% Class options include: notes, notesonly, handout, trans,
% hidesubsections, shadesubsections,
% inrow, blue, red, grey, brown

% Theme for beamer presentation.
%\usepackage{beamerthemeclassic}
% \usepackage{beamerthemebars}
% \usepackage{beamerthemesplit}
% \usepackage{beamerthemetree}
% \usepackage{beamerthemelined}

\title{Formal Modelling of a Microcontroller Instruction Set in B}

%\subtitle{Qualifica��o de Mestrado} % Enter your title betweencurlybraces
\author{Val�rio Medeiros Jr, David D�harbe}
\institute{Federal University of Rio Grande do Norte} % Enter your institute name between curly braces
\date{20 August 2009} % Enter the date or \today between curly braces

\begin{document}

%% Keywords for the B method

\newcommand{\MACHINE}{\ensuremath{\textbf{MACHINE }}}

%\newcommand{\MACHINE}{\operatorname{\textbf{MACHINE }}} % COMANDO ORIGINAL
\newcommand{\REFINEMENT}{\operatorname{\mathbf{REFINEMENT }}}
\newcommand{\IMPLEMENTATION}{\ensuremath{\textbf{IMPLEMENTATION }}}
\newcommand{\REFINES}{\ensuremath{\textbf{REFINES }}}
\newcommand{\SEES}{\ensuremath{\textbf{SEES }}}
\newcommand{\INCLUDES}{\ensuremath{\textbf{INCLUDES }}}
\newcommand{\IMPORTS}{\ensuremath{\textbf{IMPORTS }}}
\newcommand{\SETS}{\ensuremath{\textbf{SETS }}}
\newcommand{\CONSTANTS}{\ensuremath{\textbf{CONSTANTS }}}
\newcommand{\PROPERTIES}{\ensuremath{\textbf{PROPERTIES }}}
\newcommand{\CONCRETE}{\ensuremath{\textbf{CONCRETE }}}
\newcommand{\VARIABLES}{\ensuremath{\textbf{VARIABLES }}}
\newcommand{\ASSERTIONS}{\ensuremath{\textbf{ASSERTIONS }}}
\newcommand{\CONCRETEVARIABLES}{\ensuremath{\textbf{CONCRETE\_VARIABLES }}}
\newcommand{\DEFINITIONS}{\ensuremath{\textbf{DEFINITIONS }}}
\newcommand{\VAR}{\ensuremath{\textbf{VAR }}}
\newcommand{\IN}{\ensuremath{\textbf{IN }}}
\newcommand{\INVARIANT}{\ensuremath{\textbf{INVARIANT }}}
\newcommand{\INITIALISATION}{\ensuremath{\textbf{INITIALISATION }}}
\newcommand{\OPERATIONS}{\ensuremath{\textbf{OPERATIONS }}}
\newcommand{\BEGIN}{\ensuremath{\textbf{BEGIN }}}
\newcommand{\END}{\ensuremath{\textbf{END }}}
\newcommand{\PRE}{\ensuremath{\textbf{PRE }}}
\newcommand{\IF}{\ensuremath{\textbf{IF }}}
\newcommand{\THEN}{\ensuremath{\textbf{THEN }}}
\newcommand{\ELSE}{\ensuremath{\textbf{ELSE }}}
\newcommand{\ELSIF}{\ensuremath{\textbf{ELSIF }}}
\newcommand{\ANY}{\ensuremath{\textbf{ANY }}}
\newcommand{\WHERE}{\ensuremath{\textbf{WHERE }}}
\newcommand{\CASE}{\ensuremath{\textbf{CASE }}}
\newcommand{\OF}{\ensuremath{\textbf{OF }}}
\newcommand{\EITHER}{\ensuremath{\textbf{EITHER }}}
\newcommand{\AND}{\ensuremath{\textbf{AND }}}
\newcommand{\OR}{\ensuremath{\textbf{OR }}}
\newcommand{\NOT}{\ensuremath{\textbf{NOT }}}
\newcommand{\WHILE}{\ensuremath{\textbf{WHILE }}}
\newcommand{\DO}{\ensuremath{\textbf{DO }}}
\newcommand{\VARIANT}{\ensuremath{\textbf{VARIANT }}}
\newcommand{\FALSE}{\ensuremath{\textbf{FALSE }}}
\newcommand{\TRUE}{\ensuremath{\textbf{TRUE }}}

%% Commonly used math entities
\newcommand{\pow}{\ensuremath{\textbb{P }}}
\newcommand{\nat}{\ensuremath{\textbb{N }}}
\newcommand{\pfun}{\ensuremath{\rightarrow\mkern-22mu+}}
\newcommand{\fset}{\ensuremath{\textbb{F }}}
\newcommand{\dom}{\ensuremath{\mbox{dom}}}
\newcommand{\ran}{\ensuremath{\mbox{ran}}}
\newcommand{\natone}{\ensuremath{\textbb{N}_1}}
\newcommand{\integer}{\ensuremath{\textbb{Z }}}
%\newcommand{\fun}{\ensuremath{\rightarrow}}
\newcommand{\domr}{\ensuremath{\triangleleft}}
\newcommand{\seq}{\ensuremath{\textbf{seq1 }}}
\newcommand{\ovr}{\ensuremath{\oplus}}
\newcommand{\BOOL}{\ensuremath{\textbf{BOOL }}}
\newcommand{\pred}{\ensuremath{\textbf{pred }}}
\newcommand{\Bsucc}{\ensuremath{\textbf{succ }}}


\include{SymbolsB_AtelierB}

\beamertemplatefootpagenumber % Mostra o n�mero de p�ginas

% Creates title page of slide show using above information
\begin{frame}
   \titlepage  
  \begin{center} 

  Project: B2ASM \\
  \includegraphics[width=.14\textwidth]{figuras/ufrn.png} \ \ \ \ \ \ \ \ \ \
  \ \ \ \ \ \ \ \ \ \ \ \ \ \ \ \ \ \ \ \ \ \ \ \ \ \ \ \  
   \includegraphics[width=.12\textwidth]{figuras/dimap.png}
  \end{center}
  
\end{frame}
\note{Slide Principal} % Add notes to yourself that will be displayed when
  % typeset with the notes or notesonly class options


%\section [Agenda]{Agenda}


% Creates table of contents slide incorporating
% all \section and \subsection commands
 \begin{frame}
  \frametitle{Agenda}
  \tableofcontents
 \end{frame}

% Mostra um slide com destaque a atual se��o, em cada mudan�a de se��o
\AtBeginSection[]
{
\begin{frame}
\frametitle{Summary}
\tableofcontents[currentsection]
\end{frame}
}

 \include{content/introduction}
 % * Explain the cover of B Method 
 % The solution is a framework proposed a paper [Dantas, 2009]
 % The component key to this aproach is the formal model
 % Then this paper gave u

 \include{content/metodoB}
 % 

 \include{content/Modelling}

 \include{content/Description}

 \include{content/Proofs}
 
 \include{content/Related_works}

  \include{content/conclusions}



% 
% 
% \begin{frame}
% \frametitle{Refer�ncias} % Insert frame title between curly braces
%  
% \begin{itemize}
% \item J. R. Abrial. ``The B Book: Assigning Programs
 % to Meanings''. 1. ed. United States of America:
% Cambridge University Press, 1996.
% 
% \item D. Bell, Ian Morrey e John Pugh, ``Software Engineering: A Programming
% Approach", segunda edi��o, Prentice Hall, Nova Iorque, 1992.
 % 
% \item DANTAS, B. P. et al. ``Applying the b method to take on the grand
% challenge of verified compilation''. In: SBMF. \textit{Brazilian Symposium on Formal Methods}.
% Salvador - BA, 2008.  
 % 
% \item DANTAS, B. P. et al, ``Proposta e Avalia��o de uma Abordagem de Desenvolvimento de Software
% Fidedigno por Constru��o com o M�todo B'', SEMISH, 2008, B�lem.
%  
% \end{itemize}
% 
 % \end{frame}
% \note{Referencias} 
% 
% 
% \begin{frame}
% \frametitle{Refer�ncias} % Insert frame title between curly braces
% \begin{itemize}
% \item Microchip. PIC16C432 - ``Data Sheet OTP 8-Bit CMOS MCU with LIN
 % Transceiver''. USA, 2002. 197p.
% \item D.E.C. Nicolosi, ``Microcontrolador 8051 detalhado'' ,Ed. Erica, S�o
% Paulo, 2005.
% \item B. William, ``Z80 microcomputer design projects'',Ed. Sams, 1980.
 % \item UK COMPUTING RESEARCH COMMITTEE.``The Verifying Compiler: A GrandChallenge for Computing Research''.
% In UK: Workshop on Grand Challenges for Computing Research, 2005. Dispon�vel
% em:http://www.nesc.ac.uk/esi/events/Grand Challenges/workshop02.html Acessado em:18 abr 2009.  
 %  
% \end{itemize}
%  
% \end{frame}
% \note{Referencias}
%  
  
   



\end{document}